\section{Group}

\begin{definition}[amalgamated product]
  Let $\Gamma_i$ be a group and $\Lambda$ be a group with homomorphisms $\varphi_i:\Lambda$ into each $\Gamma_i$.
  We denote by $N$ a normal subgroup generated by $\varphi_1(\gamma) \varphi_2(\gamma)^{-1}$.
  $\Gamma_1 * \Gamma_2 / N$ is called amalgamated product.
\end{definition}

\begin{definition}[amalgamated free product]\label{amalgam}
  Let $\Gamma_i$ be a group and $\Lambda$ be a common subgroup (i.e. $\Lambda$ come with an injective homomorphism $\varphi_i :\Lambda$ into each $\Gamma_i$).
  Then, the amalgamated free product $\Gamma = \Gamma_1 *_{\Lambda} \Gamma_2$ is the group satusfyubg the following properties:
  \begin{itemize}
    \item $\Gamma$ contains $\Gamma_1$ and $\Gamma_2$ as subgroups and $\Gamma$ is generated by $\Gamma_1$ and $\Gamma_2$;
    \item $\Gamma_1 \cap \Gamma_2 = \Lambda$ in $\Gamma$;
    \item $s_1 \cdots s_n a \neq e $ whenever $n \geq 1$, $a \in \Lambda$ and $s_k \in \Gamma_{i_k}\backslash \Lambda$ with $i_k \neq i_{k+1}$ for $1 \leq k < n$;
    \item if we choose systems $S_i$ of represetatives of $\Gamma_i/\Lambda$ and let $S_i^0 = S_i \backslash \{e\}$ (we always assume that the representative of the coset $\Lambda$ is $e$), then any element $s$ in $\Gamma$ can be uniquely written as $s =s_1 \cdots s_na$, where $a \in \Lambda$ and $s_k \in S_{i_k}^0$ such that $i_k \neq i_{k+1}$ for $1 \leq k < n$:
  \end{itemize}
  
\end{definition}

\begin{remark}
  $\Gamma_1 *_{\Lambda} \Gamma_2 = \Gamma_1*\Gamma_2/($the smallest normal subgroup containig $\varphi_1(\lambda)^{-1}\varphi_2(\lambda))$.
\end{remark}

\begin{remark}[universality]
  If
  \[
  \xymatrix{
    N \ar[r]^{\varphi_1} \ar[d]^{\varphi_2} \ar@{}[dr]|\circlearrowleft& \Gamma_1 \ar[d]^{f_1} \\
    \Gamma_2 \ar[r]^{f_2} & \Gamma ,
  }
  \]
  then,
  \[
  \xymatrix{
    N \ar[rr]^{\varphi_1} \ar[dd]^{\varphi_2} \ar@{}[dr]|\circlearrowleft& & \Gamma_1 \ar[dd]^{f_1} \ar[dl]^{\iota_1} \\
    & \Gamma_1 *_N \Gamma_2 \ar[dr]^{ \exists f} \ar@{}[d]|\circlearrowleft \ar@{}[r]|\circlearrowleft&\\
    \Gamma_2 \ar[rr]^{f_2} \ar[ru]^{\iota_2} & & \Gamma .
  }
  \]
\end{remark}


\begin{proposition}
  \label{subgr}
  Let $G$ be a group and $H$ be a subgroup. \\
  Then, for $g_1,g_2 \in G$, $g_1H=g_2H \Leftrightarrow g_1 H \cap g_2 H \Leftrightarrow g_1^{-1}g_2 \in H$.
\end{proposition}

\begin{theorem}
  Let $G$ be a group and $A$ be a subgroup of $G$.
  Let $\theta : A \rightarrow G$ be a injective homomorphism.
  Then, there exist a group $G'$ which is generated by $G$ and $s$ s.t. $\theta(a) = sas^{-1}$ $(a \in A)$.
\end{theorem}

\begin{proof}
    \[
  \xymatrix{
    \cdots G  &  & G & & G \cdots\\
   \cdots & A \ar[lu]^\theta \ar@{^{(}-_>}[ur] &  &A \ar[lu]^\theta \ar@{^{(}-_>}[ur] & \cdots.
  }
  \]
  Amlgamation is as above.
  Let $G_n = G$ $(n \in \Z)$.
  Let $u : G_n \rightarrow G_{n+1}$ be a canonical isomorphism.
  Let $H_n := {*_A}_{k=-n}^n G_k$ $(n \in \N)$.
  We define $\varphi_n : H_n \rightarrow H_{n+1}$ by $g \mapsto 1*1*u(g)$.
  \[
  \xymatrix{
    H_1 \ar[r] \ar[dr]^{\varphi_1} & H_2 \ar[r] \ar[dr]^{\varphi_2} & H_3 \ar[r] \ar[dr]^{\varphi_2} & \cdots\\
    H_1 \ar[r] & H_2 \ar[r]  &H_3 \ar[r] & H_4 \ar[r] & \cdots.
  }
  \]
  Since the above diagram is commutative, there exist a $\sigma : *_A G \rightarrow *_A G$ s.t. $\sigma(a) = \theta(a)$ $(a \in A)$.
  Since $\Z$ acts on $*_AG$ by $n \mapsto \sigma^n$, $G':= *_A G \rtimes \Z$.
  Actually,
  \begin{align*}
    (e,1)(a,0)(e,1)^{-1} = (e,1)(a,-1) = (\sigma(a),0) = (\theta(a),0).
  \end{align*}
\end{proof}

\begin{definition}
  The above $G'$ is called the HNN extension of $G$ relative to $\theta$ and denoted by $G*_\theta$.
\end{definition}

\begin{remark}\label{hnniso}
  In the sense of the above construction, $G*_\theta = *_A G \rtimes \Z \cong G*\Z/(sas^{-1}=\theta(a))$, where $s$ is a generator of $\Z$.
\end{remark}

\begin{proof}
  A homomorphism  $G*\Z/(sas^{-1}=\theta(a)) \rightarrow *_A G \rtimes \Z$ is definef by the universality. \\
  A homomorphism  $ \varphi : *_A G \rtimes \Z \rightarrow G*\Z/(sas^{-1}=\theta(a))$ is defined by
  \begin{align*}
    (g_1 g_2 \cdots g_n , r) \mapsto k_1 g_1 (k_2 -k_1) g_2 \cdots (k_n - k_{n-1}) g_n (r - k_n),
  \end{align*}
  where $g_i \in G_{k_i}$.
  \begin{align*}
    \varphi (g_1 g_2 \cdots g_n , r) \varphi (h_1 h_2 \cdots h_m , s)
    &=  k_1 g_1 (k_2 -k_1) g_2 \cdots (k_n - k_{n-1}) g_n (r - k_n + l_1) h_1\\
    &\quad \cdot (l_2 -l_1) h_2 \cdots (h_m - h_{m-1}) h_m (s - l_m). \\
    \varphi ((g_1 g_2 \cdots g_n ,r)(h_1 h_2 \cdots h_m , s))
    &= \varphi (g_1 g_2 \cdots g_n r(h_1 h_2 \cdots h_m) , r + s) \;
    \mbox{$(r(h_i) \in G_{l_i + r} )$}\\
    &= k_1 g_1 (k_2 -k_1) g_2 \cdots (k_n - k_{n-1})  g_n ((l_1 + r) -k_n) h_1 \\
    &\quad \cdot ((l_2 + r) - (l_1 + r)) h_2 \cdots h_{m-1} (s + r - (l_m + r)) \\
    &=  k_1 g_1 (k_2 -k_1) g_2 \cdots (k_n - k_{n-1}) g_n (r - k_n + l_1) h_1\\
    &\quad \cdot (l_2 -l_1) h_2 \cdots (h_m - h_{m-1}) h_m (s - l_m). \\
  \end{align*}
  By the above calculation, the definition is well-defined.
\end{proof}

