\section{Trees}

\begin{definition}
  A graph $\Gamma$ consists of a set $Vert\Gamma$, a set $Y = Edge\Gamma$ and two maps $Y \rightarrow X \times X$, $y \mapsto (o(y), t(y))$ and $Y \rightarrow Y$, $y \mapsto \overline{y}$ (inversion) satisfying $y = \overline{\overline{y}}$, $ y \neq \overline{y}$ and $o(y) = t(\overline{y})$.
\end{definition}

\begin{definition}
  Let $T$ be a tree i.e. a connected graph with no loop. 
  \begin{itemize}
    \item For vertices $x , y \in V(T)$, we define a distance $d(x,y)$ by a length of a geodesic path (i.e. a path with no loop) from $x$ to $y$.
    \item A map $\varphi : T \rightarrow T$ is called automorphism if it is isometric bijective and we denote $Aut(T)$ by all automorphisms.
    \item For $x,y \in V(T)$, $d(x,y)=$  the length of the geodesic from $x$ to $y$. We define the topology on $T$ by this distance. 
    \item Two geodesic paths $(x(n))_n$, $(y(n))_n$ are equivalent if there is $m_0, n_0 \in \N$ s.t. $x(n+n_0) = y(n + m_0)$.
    \item The ideal boudary $\d T$ of $T$ is defined as the set of all equivalent classes of infinite geodesic paths.
    \item For $x \in \overline{T} := T \cup \d T $ and a finite set $ F \subset T$, we define $U_{x, F} := \{ y \in \overline{T} | [x,y] \cap F = \emptyset\}$. $\{U_{x,F}\}$ is a basis of the topology on $\overline{T}$.
  \end{itemize}
\end{definition}

\begin{remark}
  \begin{itemize}
    \item $\overline{T}$ is compact.
    \item $Aut(T)$ is totally desconnected.
  \end{itemize}
\end{remark}

\begin{definition}
  Let $G \curvearrowright X$ with no inversion($\Rightarrow G\cdot y \neq \overline{G \cdot y}$).
  Let $V(G \backslash T) = G \backslash V(X)$, $E(G \backslash X) = G \backslash E(X)$, $V(G \backslash E(X)) \rightarrow V(G \backslash X) \oplus  V(G \backslash X)$ $G \cdot y \mapsto (G \cdot o(y), G \cdot t(y)))$, and $E(G \backslash X) \rightarrow E(G \backslash X)$ $G \cdot y \mapsto \overline{G \cdot y}$.
  Then, $G \backslash X = (V(G\backslash X),E(G \backslash X))$ is a graph.
\end{definition}

 As below, we assume $G \curvearrowright X$ with no inversion.

\subsection{amalgamated products and a seqment} 
 
\begin{theorem}
  Let $X$ be a graph, which is $G \backslash X \cong$ a segment $T=(\{P,Q\}, \{y,\overline{y}\}) < X$.
  Then, the followings are equivalent.
  \begin{itemize}
    \item $X$ is a tree;
    \item A homomorphism $\varphi : G_P *_{G_y}G_Q \rightarrow G$ induced by $G_P (G_Q) \hookrightarrow G_y$ is an isomorphism.
  \end{itemize}  
\end{theorem}

\begin{remark}
  There is no element $g \in G$ s.t. $g P = Q$.
  Indeed, if not, $V(G\backslash X)$ is a point.
\end{remark}

This theorem follows from the two following lemmas.

\begin{lemma}
  $X$ is connected iff $\varphi$ is surjective.
\end{lemma}

\begin{proof}
  Let $X'$ be a connected component of $X$ containig $T$.
  Let $G' := \{ g \in G \;|\; g X' = X'\}$.
  Let $G''$ be a subgroup generated by $G_P$ and $G_Q$.
  We prove $G' = G''$.\\
  For $ g \in G_P \cup G_Q$, $y$ and $gy$ have a common vertex.
  So, $g X' = X'$.
  Thus, $G'' \subset G'$.\\
  Since $G''T$ and $(G-G'')T$ are disjoint, $X' \subset G''T$.
  For $g \in G'$, $gT \subset gX' \subset G''T$,
  so there exists a $h \in G''$ s.t. either $gP=hP$ or $gQ =hQ$ by the above remark.
  we may assume $gP = hP$.
  Then, $h^{-1}g P = P$, so $h^{-1}g \in G_P$.
  Hence, $g = h h^{-1} g \in G''$.
  So, $G' \subset G''$.
  Therefore, $G' = G''$. \\
  $X$ is connected $\Leftrightarrow$ $X = X'$ $\Leftrightarrow$ $G = G'$ $\Leftrightarrow$ $G = G''$.  
\end{proof}

\begin{lemma}
  $X$ has no loop iff $\varphi$ is injective.
\end{lemma}

\begin{proof}
  Assume $X$ has a loop $c := (c_1 , \ldots , c_n )$.
  We may assume $c_1 = y$.
  Let $P_k := t(c_k)$ $(k \geq 1)$.
  There exist a $g_k \in G$ and a $y_i \in \{y, \overline{y}\}$ s.t. $c_k = g_k y_k$
  By the above remark,
  $y_{k+1} = \overline{y_k} $. 
  Hence, $h_k := g_{k+1}^{-1}g_k \in G_{t(y_k)}$.
  If $h_k \in G_y$, it is contradiction to $y_{k+1} = \overline{y_k}$.
  So, $h_k \notin G_y$.
  $P = t(c) = t(g_n y_n) = g_1 h_1^{-1} \cdots h_n^{-1} P$.
  Since $g_1 = 1$, $h_1^{-1} \cdots h_n^{-1} \in G_P$.
  There exists a $k \in G_P$ s.t. $k h_1 \cdots h_n =1$.
  This is contradiction to $k h_1 \cdots h_n \neq 1$, by the definition \ref{amalgam} of amalgamated free product. 
\end{proof}

\subsection{Amalgamated products and trees}

\begin{definition}
  A graph of groups is consisting of a connected nonempty graph $Y$ and groups $G_P$ $(P \in V(Y))$ $($resp.$G_y$ $(y \in E(Y)))$, and denoted by $(G,Y)$ s.t. $G_y \rightarrow G_{t(y)}$, which is denoted by $a \mapsto a^y$, and $G_y = G_{\overline{y}}$.\\
  For a graph of groups $(G,Y)$, we denote $\lim (G,Y)$ by $G_T$.
\end{definition}

\begin{theorem}
  Let $(G,Y)$ be a graph of groups.
  There exists a graph $X$ on which $G_T$ acts s.t. $T$ is a fundamental domain of $X$ with respect to $G_T$ and $(G_T)_P = G_P (P \in V(T)) ($resp. $(G_T)_y=G_y (y \in E(Y)))$. \\
  Moreover, $X$ is a tree.
\end{theorem}

\begin{theorem}
  Let $G$ be a group acting on a graph $X$.
  Let $T$ be a tree whose fundamental domain with respect to $G$ is a tree $T$.
  Let $\varphi : G_T \rightarrow G$ be a homomorphism induced by $G_P \rightarrow G$, which is surjective, since $X$ is connnected.
  Let $\psi:\tilde{X} \rightarrow X$ be a homomorphism which is uniquely determined by $T \rightarrow T$ and $\varphi$.
  The followings are equivalent.
  \begin{itemize}
    \item $X$ is a tree.
    \item $\psi$ is an isomorphism.
    \item $\varphi$ is an isomorphism.
  \end{itemize}
\end{theorem}

\subsection{Fundamental groups of a graph of groups}

\begin{definition}
  Let $(G,Y)$ be a graph of groups.\\
  We denote $F(G,Y)$ by the quotient of $\ast_{P\in V(T)}\ast F$ by the normal subgroup generated by $y\overline{y}$ and $ya^yy^{-1}{(a^{\overline{y}})}^{-1} (y \in E(Y),a \in G_{y})$, where $F$ is a free group generated by $E(Y)$.\\
  For $P_0 \in V(Y)$,
  \begin{align*}
    \pi_1(G,Y,P_0) := \left\{ |c,\mu| := r_0y_1r_1\cdots y_nr_n  \;\middle|\;{\begin{gathered} c = (y_1,y_2, \ldots , y_n)\; is\; path\; in\; Y\;\\
    s.t.\, o(y_1)=t(y_n), r_i \in G_{t(y_i)}\end{gathered}}\right\}.
  \end{align*}
  For a maximal tree $T < Y$, we denote $\pi_1(G,Y,T)$ by the quotient $F(G,Y)$ by the normal subgroup generated by $y=1$ $(y \in E(T))$.
  Then,we denote $g_y$ by the image in $\pi_1(G,Y,T)$ of $y \in E(Y)$.
\end{definition}

\begin{remark}
  $F(G,Y)$ has the relation $ya^y y^{-1} =a^{\overline{y}}$ $(y \in E(Y), a \in G_y)$.
  $\pi_1(G,Y,T)$ has the above relation and the relation $g_y=1$ $y \in E(T)$.
\end{remark}

\begin{remark}
  $\pi_1(G,Y,P_0)$ and $\pi_1(G,Y,T)$ are isomorphic, so denoted by $\pi_1(G,Y)$.
  This follows from the following proposition.
\end{remark}

\begin{lemma}
  $p : F(G,Y) \rightarrow \pi_1(G,Y,T)$ induces an isomorphism $\overline{p} : \pi_1(G,Y,P_0) \rightarrow \pi_1(G,Y,T)$.
\end{lemma}

\begin{proof}
  For $P \in V(T)$, we define $c_P = (y_1, \ldots, y_n)$ by the path from $P_0$ to $P$ with no backtracking
  and $\gamma_P := y_1 \cdots y_n$.
  For $x \in G_P$ and $y\in E(Y)$, let $x' := \gamma_P x {\gamma_P}^{-1} \in \pi_1(G,Y,P_0)$ and
  $y' := \gamma_{o(y)}y{\gamma_{t(y)}}^{-1} \in \pi_1(G,Y,P_0)$. \\
  Then, for $y \in E(T)$, either $c_{t(y)} = (c_{o(y)},y)$ or $c_{o(y)} = (c_{t(y)},\overline{y})$ is satisfied, so $y' =1$.
  Also, $\overline{y}' = \gamma_{o(\overline{y})}\overline{y}{\gamma_{t(\overline{y})}}^{-1} = \gamma_{t(y)}\overline{y}{\gamma_{o(y)}}^{-1}$,
  so $y'\overline{y}' = \overline{y}'y' =1$.
  For $y \in E(Y)$ and $a \in G_y$,
  \begin{align*}
    y' (a^y)'{y'}^{-1}
    &= \gamma_{o(y)} y {\gamma_{t(y)}}^{-1} \gamma_{t(y)} a^y {\gamma_{t(y)}}^{-1}  \gamma_{t(y)} y^{-1} {\gamma_{o(y)}}^{-1} \\
    &=  \gamma_{o(y)} a^{\overline{y}} {\gamma_{o(y)}}^{-1} \\
    &= (a^{\overline{y}})'.
  \end{align*}
  So, there exists a unique homomorphism $f: \pi_1(G,Y,T) \rightarrow \pi_1(G,Y,P_0)$ s.t. $f(\overline{x})= x'$ and $f(\overline{y}) = y'$. \\
  Since $p\circ f = \id$, $\overline{p}$ is injective.
  By the construction, $\overline{p}$ is surjective.   
\end{proof}
  

\begin{example}
  If $G_y = \{e\}$, $\pi_1(G,Y,T) = \ast_{P \in V(Y)}G_P \ast F$.
\end{example}

\begin{example}
  If $Y$ is a seqment, $\pi_1(G,Y,T) = G_P \ast_{G_y} G_Q$.
\end{example}

\begin{example}
  If $Y$ is a loop, $\pi_1(G,Y,T) = $HNN-extension.
  It is generated by $G_P$ and $g_y$, and satisfies the relation $g_y a^y g_y^{-1} = a^{\overline{y}}$
\end{example}

\begin{definition}
  Let $(G,Y)$ be a graph of groups and $X < Y$ be a (connected) subgraph. \\
  We define $(G,Y/X)$ as below. \\
  $V(Y/X) := V(Y) \sqcup \{V(X)\}$, $E(Y/X) := E(Y) \backslash E(X)$ and
  \begin{align*}
  o(y) := \left\{\begin{gathered} o(y)\;\;  (o(y) \notin V(X))\\
   V(X) \;\; (o(y) \in V(X)) \end{gathered} \right.
  , o(y) := \left\{\begin{gathered} t(y)\;\;  (t(y) \notin V(X))\\
   V(X) \;\; (t(y) \in V(X)). \end{gathered} \right.
  \end{align*}
  $G_{\{X\}}:= \pi_1(G,X)$.
\end{definition}

\begin{remark}
  $\pi_1(G,Y) = \pi_1(G,Y/X)$.
\end{remark}

\begin{remark}[cf. Rem.\ref{hnniso}]
  Let $(G,Y)$ be a graph of groups.
  Let $T < Y$ be a maximal tree in $Y$.\\
  Then, $Y' := Y/T$ is an order-$\rm{rank}(\pi_1(Y))$ bouquet graph and $\pi_1(Y) \cong \pi_1(Y')$. \\
  Let $\tilde{T}$ be a universal covering of $Y'$ and its covering map is denoted by $\pi$.\\
  We define $(G,\tilde{T})$ by
  $G_{\tilde{P}} := G_{\pi(\tilde{P})}$ $(\tilde{P} \in V(\tilde{T}))$
  $($resp. $G_{\tilde{y}} := G_{\pi(\tilde{y})}$ $(\tilde{y} \in E(\tilde{T}))$.\\
  Then, $\pi_1(Y')$ acts on $\tilde{T}$ by the deck transformation $Deck(\tilde{T}/Y') \cong \pi_1(Y')$,
  so $\pi_1(Y')$ acts on $\pi_1(G,\tilde{T})$.\\
  We define $\Theta:\pi_1(G,Y')\rightarrow \pi_1(G,\tilde{T}) \rtimes \pi_1(Y')$
  by
  \begin{align*}
    \pi_1(G,Y',P_0) &\rightarrow \pi_1(G,\tilde{T},\tilde{T}) \rtimes \pi_1(Y'), \\
    r_0 y_1 r_1 \cdots y_n r_n &\mapsto (r_0 \cdots r_n , y_1 \cdots y_n), 
  \end{align*}
  where $Q_0 \in V(\tilde{T})$ and $r_0 \in G_{Q_0}, r_1 \in G_{y_1 (Q_0)}, r_2 \in G_{y_1 y_2 (Q_0)}, \cdots, r_n \in G_{y_1 \cdots y_n (Q_0)}$. 
  $\Theta$ is isomorphism.\\
  We define a homomorphism $\pi_1(G,\tilde{T},\tilde{T}) \rtimes \pi_1(Y') \rightarrow \pi_1(G,Y',P_0)$ by
  \begin{align*}
    (r_0 \cdots r_n , y) \mapsto r_0 \pi(\gamma_{1}) r_1 \cdots \pi(\gamma_n) r_n \pi(\zeta),
  \end{align*}
  where for $r_i \in G_{P_i}$ $(P_0 = Q)$, $\gamma_i$ is a geodedic from $P_i$ to $P_{i+1}$ in $T$ and $\zeta$ is a geodedic from $P_n$ to $Q$ in $T$.
  In conclusion, $\pi_1(G,Y) = \pi_1(G,Y') \cong \pi_1(G,\tilde{T}) \rtimes \pi_1(Y)$.
\end{remark}


\begin{definition}
  Let $(G,Y)$ be a graph of groups.
  For the path $c := (y_1, \ldots, y_n)$, and $\mu = (r_0, \ldots, r_n)$ $(a_i \in G_{t(y_i)}$, $r_0 \in G_{o(y_1)})$,
  we define
  \begin{align*}
    |c,\mu| := r_0 y_1 r_2 \cdots r_n y_n
  \end{align*}

  One says that $(c,\mu)$ is reduced if it satisfies the following condition: 
  If $n=0$ one has $r_0 \neq 0$; if $n \leq 1$ one has $r_i \notin G_{y_i}^{y_i} := \rm{Im}(G_{y_i} \rightarrow G_{t(y_i)})$ for each index $i$ s.t. $y_{i+1} = \overline{y_i}$. 
\end{definition}

\begin{theorem}
  If $(c,\mu)$ is a reduced word, the associated element $|c,\mu|$ of $F(G,Y)$ is $\neq 1$.
\end{theorem}

The following corollary follows from $\pi_1(G,Y,P_0) \cong \pi_1(G,Y,T)$. 

\begin{corollary}
  Let $T < Y$ be a maximal tree and let $(c,\mu)$ be a reduced word whose type $c$ is a closed path.
  Then, $\overline{|c,\mu|} \neq 1$ in $\pi_1(G,Y,T)$.
\end{corollary}

\subsection{Universal covering relative to a graph of groups}

Let $(G,Y)$ be a graph of groups.
Let $T < Y$ be a maximal tree.
Let $A < Y$ be an orientation $($i.e. $Y = A \sqcup \overline{A})$.
For $y \in E(Y)$, 
\begin{align*}
  |y| := \left\{\begin{gathered} y\;\;  (y \in A)\\
  \overline{y} \;\; (y \notin A) \end{gathered} \right.
  , \; e(y) := \left\{\begin{gathered} 0 \;\;  (y \in A)\;\;\\
  1 \;\; (y \notin A) \;.\end{gathered} \right.
\end{align*}

We construct the following objects.
\begin{itemize}
  \item graph $\tilde{X} = \tilde{X}(G,Y,T)$;
  \item $\pi := \pi_1(G,Y,T)$ acts on $\tilde{X}$;
  \item $p:\tilde{X} \rightarrow Y$ induces an isomorphism $\pi \backslash \tilde{X} \cong Y$;
  \item sectios $V(Y) \rightarrow V(\tilde{X})$ and $E(Y) \rightarrow E(\tilde{X})$ of $p$, which is denoted by $P \mapsto \tilde{P}$ and $y \mapsto \tilde{y}$;
    \item $\pi_{\tilde{P}} = G_P$, $\pi_{\tilde{y}} = G_{\overline{|y|}}^{\overline{|y|}}$:
\end{itemize}

Let $V(\tilde{X}) := \sqcup_{P \in V(Y)}\pi/\pi_P$ and $E(\tilde{X}) := \sqcup_{y \in E(Y)} \pi/\pi_{y}$, where $\pi_P := G_P$ $(P\in V(Y))$ and $G_y := G_{\overline{|y|}}^{\overline{|y|}}$ $(y \in E(Y))$. \\

We denote the image of $1$ in $\pi/\pi_P \; ($resp. $\pi/\pi_y)$ by $\tilde{P} \; ($ resp. $\tilde{y})$. 

For $g \in \pi$ and $y \in E(Y)$,

\begin{align*}
  \overline{g \tilde{y}} &:= g \tilde{\overline{y}}, \\
  o(g\tilde{y}) &:= g g_y^{-e(y)}\tilde{o(y)}, \\
  t(g \tilde{y}) &:= g g_y^{1-e(y)}\tilde{t(y)}.
\end{align*}

Then, $\pi_y = \pi_{\overline{y}}$.
Also, for $h \in \pi_{\tilde{y}}$,
$h g_y^{-e(y)}\tilde{o(y)} = g_y^{-e(y)}\tilde{o(y)}$.

\begin{theorem}
  The above graph $\tilde{X}$ is a tree.
\end{theorem}

\begin{proof}
  Connectedness is ganbaru. \\
  We show that $\tilde{X}$ have no closed path with no backtracking.
 
\end{proof}

\subsection{ping-pong lemma}

In this section, we consider a tree as a connected set in $\R^2$.

\begin{definition}
  Let $\gamma$ be a isometric bijection of a tree $T$.
  We define $l(\gamma) := \inf_{x \in T} d(x,\alpha x)$
  \begin{itemize}
    \item If $\gamma$ fix a point in $T$, then $\gamma$ is called elliptic;
    \item If $\gamma$ does not fix any point in $T$, then $\gamma$ is called hyperbolic:
  \end{itemize}
      
\end{definition}

\begin{remark}
  For $\gamma$, $\Fix(\gamma)$ is a tree.
\end{remark}

\begin{proposition}
  Let $\gamma$ be a hyperbolic element.
  Then, there exist the unique $\gamma$-invariant line.
  We denote this line by $\Axis(\gamma)$. 
\end{proposition}

\begin{remark}
  Suppoese there exist $x \in T$ s.t. $d(x,\gamma^2 x) = 2 d(x,\gamma x)$.
  $\gamma x$ is in $[x,\gamma^2 x]$, since $\gamma$ is hyperbolic.
  $[x,\gamma x, \gamma^2 x]$ generate a $\gamma$-invariant line. 
\end{remark}

\begin{proof}
  Let $y \in T$.
  If $x \in [\gamma x, \gamma^2 x]$, then it contradicts hyperbolicity.
  If $\gamma x \in [x, \gamma^2 x]$, then it is what we want.
  So, we may asuume $x$, $\gamma x$, $\gamma^2 x$ are common points.
  we remark $\gamma x \neq \gamma^2 x$.\\
  Let $\alpha$ (resp. $\beta$) be geodesic from $x$ to $\gamma x$ (resp. $\gamma^{-1} x$).
  Let $n$ be a maximal number which satisfies $\alpha(k)=\beta(k)$ $(1 \leq k \leq n)$ and
  let $a = \alpha(n) = \beta(n)$, which is the crux of a triangle with vertecies $x$, $\gamma x$, $\gamma^{-1}x$.
  Let $b = \gamma a$ and $c = \gamma^{-1}a$, which is not $a$ by hyperbolicity.
  Then, $b$ in $[x,\gamma x]$, since $a$ in $[x,\gamma^{-1}x]$. 
  If $d(b,x) < d(x,a)$, then $x$ in $[x,\gamma x]$, so $a$, $\gamma a$, $\gamma^{-1} a$ is on the same line.
  So, we may assume $d(b,x) > d(x,a)$.
  Similarily, we may asuume $c$ in $[x,\gamma^{-1}]$ and $d(c,x) < d(x,a)$.
  So, $d(\gamma^{-1} a,\gamma a) = 2 d(\gamma^{-1} a,a)$.
\end{proof}

\begin{remark}
  For isometry $g$, $g \Axis(\gamma) = \Axis(g \gamma g^{-1})$
\end{remark}

\begin{proposition}
  Let $\gamma$, $\delta \in \Aut(T)$.
  \begin{enumerate}
  \item If $\gamma$, $\delta$ are elliptic and $\Fix(\gamma) \cap \Fix(\delta) = \emptyset$,
    then $\gamma \delta$ is hyperbolic with $l(\gamma \delta) = 2d(\Fix(\gamma),\Fix(\gamma))$.
  \item  If $\gamma$, $\delta$ are hyperbolic and $\Axis(\gamma) \cap \Fix(\delta) = \emptyset$,
    then $\gamma \delta$ is hyperbolic with $l(\gamma \delta) l(\gamma) + l(\delta) + 2d(\Axis(\gamma),\Axis(\delta))$
    and $\Axis (\gamma \delta)$ intersects $\Axis (\gamma)$ and $\Axis(\delta)$.
  \end{enumerate}
\end{proposition}

\begin{proof}
  There exist $x \in \Fix(\gamma)$ and $y \in \Fix(\delta)$ s.t. $d(x,y) = d(\Fix(\gamma),\Fix(\delta))$.
  Let $\alpha$ be a geodesic from $x$ to $y$.
  Then, $\alpha \cup \overline{\delta \alpha}$ is a geodesic from $\delta x$ to $y$.
  Also, $\delta \alpha$ is a geodesic path form $\delta \Fix(\gamma)$ to $\Fix(\delta)$.
  Indeed, if $\delta \alpha \cap \alpha = [y,z]$ is not a vertex, that is $y \neq z$, this is contradiction to $\Fix(\gamma) \cap \Fix(\delta) = \emptyset$.
  So, $d(x, (\gamma \delta)^2 x) = 2 d(x, \gamma \delta x)$.

  There exist $x \in \Axis(\gamma)$ and $y \in \Axis(\delta)$ s.t. $d(x,y) = d(\Axis(\gamma),\Axis(\delta))$.
  \begin{alignat*}{5}
    \Axis(\gamma) &\rightarrow& \Axis(\delta) &\rightarrow& \Axis(\delta) &\rightarrow& \gamma \Axis(\delta) &\rightarrow& \gamma \Axis(\delta) &\rightarrow \delta \gamma \Axis(\delta) = \delta \Axis(\gamma)\\
    x &\rightarrow& y &\rightarrow& \gamma y &\rightarrow& \gamma x &\rightarrow& \delta \gamma x &\rightarrow \delta \gamma y
  \end{alignat*}
  So, $d(x,(\delta \gamma)^2 x) = 2d(x,\delta \gamma x)$. 
\end{proof}

\begin{lemma}
  Let $e$, $e' \in E(T)$ and an isometry $\gamma$ s.t. $\gamma e = e'$.
  Let $x = o(e)$ and $y = t(e)$. 
  If $d(x, \gamma x) = d(y, \gamma y)$, then $\gamma$ is hyperbolic.
\end{lemma}

\begin{proof}
  ?
\end{proof}

\begin{lemma}[ping-pong lemma]
  Let $\gamma$, $\delta$ be hyperbolic elements whose axes has a intersection.
  If this intersection is compact, there exist $n \in \N$ s.t. $\gamma^n$ and $\delta$ generate a free group of rank $2$.
\end{lemma}

\begin{proof}
  Let $K := \Axis(\gamma) \cap \Axis(\delta)$. $n = |K|$.
\end{proof}

\subsection{Amenabilty and hyperbolic element}

\begin{theorem}[\cite{nebbia1988amenablitz}]
  Let $T$ be a locally finite tree and $G < \Aut(T)$ be a closed subgroup.
  Then, $G$ is amenable if and only if one of the followinf statements holds
  \begin{itemize}
  \item $G$ fixes a vertex;
  \item $G$ stabilize an edge;
  \item $G$ fix a point in $\d T$;
  \item $G$ stabilize a pair of points in $\d T$:
  \end{itemize}
\end{theorem}

\begin{proposition}
  Let $G < \Aut(T)$ be a closed non-amenable subgroup.
  There exists a hyperbolic element in $G$.
\end{proposition}

\begin{lemma}
  If for any $g,h \in G$, $\Fix(g) \cap \Fix(h) \neq \emptyset$,
  then $\cap_{fin.}\Fix(g) \neq \emptyset$.
\end{lemma}

\begin{proof}
  We remark $\Fix(g)$ is a tree.
  If $\Fix(g) \cap \Fix(h) \cap \Fix(k) = \emptyset$,
  there exists a cycle.
  Contradiction.
\end{proof}

\begin{proof}[proof of proposition]
  We assume $G$ has no hyperbolic element. 
  If there exist elliptic elements $g,h \in G$ s.t. $\Fix(g) \cap \Fix(h) = \emptyset$, there exist a hyperbolic element.
  So, for any $g,h \in G$, $\Fix(g) \cap \Fix(h) \neq \emptyset$.
  Since $\overline{T}$ is a complete metric space,
  \begin{align*}
    \cap_{g \in G} \Fix_{\overline{T}}(g) \neq \emptyset.
  \end{align*}
  It is contradiction to non-amenability of $G$.
\end{proof}
