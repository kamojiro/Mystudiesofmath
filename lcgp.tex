\section{Locally compact groups}

\subsection{topological gropus}

\begin{theorem}
  Let $G$ be a connected topological group.
  Let $U$ be a neighborhoff of $e$ s.t. $U = U^{-1}$.
  Then, for any $ g \in G$, there exists $g_1, \ldots , g_k$ s.t. $g = g_1 \ldots g_k$.
\end{theorem}

\begin{theorem}
  Let $G$ be a compact group acting on topological space $X$.
  Then, a quotient map $\pi:X \rightarrow G \backslash X$ is proper (i.e. $\pi^{-1}(cpt.)=cpt.$).
\end{theorem}

\begin{proof}
  Let $F \subset X \backslash G$ be a compact subset.
  Let $\{U_i\}_{i \in I}$ be a open covering of $F$.\\
  For $ x \in \pi^{-1}(F)$, there exist a $i_x \in I$ s.t. $ \pi(x) \in U_{i_x}$.
  For $g \in G$, there exist a open neighborhood $V_{x,g} \subset G$ of $g$ and a open neighborhood $W_{x,g}$ of $x$ s.t. $V_{x,g}\cdot W_{x,g} \subset U_{i_x}$.\\
  By compactness of $G$, $G = \cup_{fin.}V_{x,g_n}$.
  Let $W_x := \cap W_{x,g_n}$.
  Then, $G\cdot W_x \subset U_{i_x}$.\\
  Since $F \subset \cup_{x \in \pi^{-1}(F)} \pi(G W_x)$ and $G\cdot W_x$ is open, $F \subset \cup_{fin.} \pi(G\cdot W_x)$. \\
  So, $\pi^{-1}(F) \subset \cup_{fin.}G\cdot W_x \subset \cup_{fin.} U_{i_x}$.
\end{proof}

\begin{proposition}
  Let $G$ be a topological group and $H$ be a topological subgroup of $G$ with finite index.
  $G$ is topologically finitely generated if ond only if so is $H$.
\end{proposition}

\begin{proof}
  We assume $H$ is topologically finitely generated.
  Let $F$ be a fintie generator of $H$.
  Then, $G = \sqcup_{fin.} g_i H$.
  So, $F \cup \{g_i\}$ is a finite generator of $G$.


  We assume $G$ is topologically finitely generated.
  Let $Y$ be a finite generator of $G$.
  Let $[\cdot] : G \rightarrow G$ $g \mapsto [g]$ be a left $H$-invariant map.
  We may assume $e = [e]$.
  We have, for any $g \in G$
  \begin{align*}
    g[g]^{-1} \in H, \; [[g]h] = [gh], \; [g] = [[g]].
  \end{align*}
  $T = \{[g]y[[g]y]^{-1} | g \in G,\; y \in Y\}$ is finite and generator of $H$.
  Finiteness is follows from finite index.
  Indeed, suppose $h = g_1 g_2 \cdots g_r \in H$ ($g_i \in Y$).
  For simplicity, $r = 3$.
  \begin{align*}
    h = g_1 [g_1]^{-1} \cdot [g_1] g_2 [[g_1] g_2]^{-1} \cdot [[g_1] g_2] g_3 [[[g_1] g_2] g_3]^{-1}. 
  \end{align*}
  Each element belongs to $T$. 
\end{proof}

\subsection{locally compact group}

Let $G$ be a locally comapct group.
Let $\mu (\; =\mu_G)$ be a left Haar measure on $G$. 
Let $\bigtriangleup (\, = \bigtriangleup_G)$ be a modular function on $G$.  

\begin{corollary}
  $H$ be a compact subgroup.
  A quotient map $\pi : G \rightarrow G/H$ is proper.
\end{corollary}

\begin{proposition}
  Then, $G$ is tottaly disconnected iff $G$ has a fundamental system of neighborhoods for a unit $e$ consistinng of compact open subgroups.
\end{proposition}

\subsection{Haar measure}

\begin{proposition}
  Let $K$ be a compact open subgroup of $G$.
  Then, $\ker(\bigtriangleup_G)$ is a clopen normal subgrop containig $K$. 
\end{proposition}

\begin{definition}
  We denote $\ker(\bigtriangleup_G)$ by $G_0$.
\end{definition}

\begin{proposition}
  Let $G$ be a totally disconnected. 
  Let $K$ be a compact normal subgroup of $G$.
  Let $\mu = \mu_K$.
  Then, $\mu$ is invariant under the conjugation action of $G$.
\end{proposition}

\begin{proof}
  We suffies to show that for any open subgroup $H < K$ and for any $g \in G$, $\mu(H) = \mu(gHg^{-1})$,
  since a borel set is generated by open subsets and a topological group has fundamental system of neighborhoods of $e$. \\
  Suppose $\mu (H) < \mu(gHg^{-1})$.
  $K = \sqcup_{fin.}k_i H$, since $H$ is a open subgroup.
  \begin{align*}
    \mu(K) = \sum_i \mu(H)
    < \sum_i \mu(gHg^{-1})
    = \sum_i \mu((g k_i g^{-1})g H g^{-1})
    = \mu(gKg^{-1})
    = \mu(K), 
  \end{align*}
  which is contradiction.
\end{proof}

\begin{remark}
  This proposition is satisfied in the case of general locally compact groups.
\end{remark}

\begin{proof}
  For $g \in G$, $\mu_g := \mu(g \cdot g^{-1})$ is a left Haar measure.
  By the uniqueness of a Haar measure, $\mu = \mu_g$.
\end{proof}

\begin{proposition}
  Let $G_i$ be a locally compact group and $K$ be a common open subgroup.
  Then, $G_1 *_K G_2$ is a locally comapct group with respect to the topology generated by $K$.
  Moreover, if $G_i$ is unimodular, $G_1 *_K G_2$ is unimodular.
\end{proposition}

\begin{proof}
  Let $\mu := \mu_{G_1 *_K G_2}$.
  We suffices to show that $\mu(gEg^{-1}) = \mu(E)$ for all $g \in G_i$ and $E \subset K$,
  because $gE = \sqcup_{g_j}g_j E_i$ $(g \in G_i , \; E_i \subset K)$.
  It follows from unimodularity of $G_i$.
\end{proof}


\subsection{Amenability}

\begin{definition}
  $G$ is amenable if it has a left invarinat mean, that is a left invariant state on $\rm{L}^{\infty}(G,\mu)$.
\end{definition}

\begin{proposition}
  Let $G$ be a group acting on $(X, \nu)$.
  Assume $X$ has a $G$-invarinat mean,
  the action is measure preserving and 
  the satabilizer subgroup $G_x$ is amenable for a.e. $x \in X$.
  Then, $G$ is amenable.
\end{proposition}

\begin{proof}
  Let $\mu_x$ be a left invarinat mean on $G_x$.
  Let $\mu_X$ be a $G$-invarinat mean on $X$.
  Let $X = \sqcup_i G\cdot x_i$.\\
  For $x \in X$, there exist a $g_x \in G$ s.t. $x = g_x x_i$.
  For, $f \in \L^\infty(G)$ and $x \in X$, 
  $\theta(f)(x) := \mu_{x_i}((g_x^{-1}. f)|_{G_{x_i}})$.
  For $g \in G_{x_i}$, $\mu_{x_i}(((g_x g)^{-1}. f)|_{G_{x_i}}) = \mu_{x_i}(g^{-1}.(g_x^{-1}.f)|_{G_{x_i}}) = \mu_{x_i}((g_x^{-1}. f)|_{G_{x_i}})$,
  so it is well-defined.
  We define $\mu$ by $\mu(f) := \mu_X(\theta(f))$.\\ 
  For $g,\; h \in G$, $\theta(h.f)(g x_i) = \mu_{x_i}(g_x^{-1}.(h.f)|_{G_{x_i}}) = \mu_{x_i}((h^{-1}g_x)^{-1}f)|_{G_{x_i}}) = (h.\theta(f))(x)$.
  So, $\mu$ is a left invariant mean. 
\end{proof}

\begin{corollary}
  Let $G$ be a locally compact group which is an extension of $N$ by $H$.
  Assume $N$ and $H$ are amenable group.
  Then, $G$ is amenable.
\end{corollary}

\begin{example}[amenale]
  \begin{itemize}
    \item abelian groups;
    \item compact groups;
    \item $\Z/2\Z * \Z/2\Z = \Z \rtimes \Z/2\Z$, since extension;
  \end{itemize}
\end{example}

\begin{example}[non-amenable]
  \begin{itemize}
    \item $\mathbb{F}_2$;
    \item $\Z/n\Z * \Z/m\Z$ $(n \geq 2, m \geq 3)$, like $\mathbb{F}_2$;
  \end{itemize}
\end{example}

\begin{theorem}[\cite{brown2008c}]
  Let $G$ be a countable discrete group.
  Then, the followings are wquivalent.
  \begin{enumerate}
  \item $G$ is amenabale;
  \item $G$ satisfies the F{\o}lner condition (i.e., for any finite subset $E \subset G$ and $\varepsilon > 0$, there exist a finite subset $F \subset G$ s.t.
    \begin{align*}
      \max_{s \in E} \frac{|sF\bigtriangleup F|}{|F|} < \varepsilon
    \end{align*}
    );
  \item the trivial representation $\tau_0$ is weakly contained in the regular representation $\lambda$ (i.e., there exist unit vectors $\xi_i \in l^2(G)$ s.t. $\|\lambda_s(\xi_i) - \xi_i\| \rightarrow 0$ for all $s\in G$);
  \item there exists a net $(\varphi_i)$ of finitely supported positive definite functions on $G$ s.t. $\varphi_i \rightarrow 1$ pointwise;
  \item $C^*(G) = C^*_r(G)$;
  \item $C^*_r(G)$ has a character (i.e., one-dimensional representation);
  \item for any finite subset $E \subset G$, we have
    \begin{align*}
      \|\frac{1}{|E|}\sum_{s \in E} \lambda_s\| = 1;
    \end{align*}
  \item $C^*_r(G)$ is nuclear;
  \item $L(G)$ is semidiscrete.
  \end{enumerate}
\end{theorem}

