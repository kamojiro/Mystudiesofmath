\section{Boundary}
We assume $G$ is discrete group.
\subsection{injective envelope}
\begin{definition}
  Let $\mathcal{S}$ be a operator system, i.e. a unital self-adjoint subspace of a unital $\rm{C}^*$-algebra.
  
  We say $G$-operator system if there is a homomorphism from $G$ into the group of order isomorphism on $\mathcal{S}$ that sends the identity element of $G$ to the unit of $\mathcal{S}$.
  
  A $G$-operator system $\mathcal{U}$ is $G$-injective if for every unital c.i. $G$-equivariant map $\iota : \mathcal{S} \rightarrow \mathcal{T}$ and
  every unital c.p. $G$-equvariant map $\psi : \mathcal{S} \rightarrow \mathcal{U}$,
  there exists unital c.p. $G$-equivariant map $\hat{\varphi} : \mathcal{T} \rightarrow \mathcal{U}$ s.t. $\hat{\varphi} \circ \iota = \psi$
  \[
  \xymatrix{
    \mathcal{S} \ar[rr]^{\iota(c.i.)} \ar[dr]_{\psi(c.p.)} &\ar@{}[d]|\circlearrowleft &\mathcal{T} \ar@{-->}[dl]^{\hat{\varphi}(c.p.)} \\
    &\mathcal{U}& .
  }
  \]
  
  A $G$-extension of $\mathcal{S}$ is a pair $(\mathcal{T},\iota)$ consisting of a $G$-operator space $\mathcal{T}$ and c.i. $G$-equivariant $\iota : \mathcal{S} \rightarrow \mathcal{T}$ .

  A $G$-extenstion $(\mathcal{U},\iota)$ is $G$-injective.

  It is $G$-essensial if for every unital c.p. $G$-equivarinat map $\varphi: \mathcal{U} \rightarrow \mathcal{T}$ s.t. $\varphi \circ \iota$ is c.i. on $\mathcal{S}$ is necessarily c.i.
  \[
  \xymatrix{
    \mathcal{S} \ar[r]^{\iota(c.i.)}&\mathcal{U} \ar[d]^{\varphi(c.p.\Rightarrow c.i.)} \\
    & \mathcal{T}.
  }
  \]

  It is $G$-rigid if for every unital c.p. $G$-equivarinat map $\varphi : \mathcal{U} \rightarrow \mathcal{U}$ s.t. $ \varphi \circ \iota = \iota$ on $\mathcal{S}$, $\varphi$ is necessarily the identity map on $\mathcal{U}$.
  \[
  \xymatrix{
    \mathcal{U} \ar[rr]^{\varphi (c.p.) \Rightarrow \rm{id}} &\ar@{}[d]|\circlearrowleft &\mathcal{U} \\
    &\mathcal{U} \ar[lu]^{\iota} \ar[ur]_\iota& .
  }
  \]

  It is $G$-injective envelope of $\mathcal{S}$ if it is $G$-injective and $G$-essential.
\end{definition}

\begin{remark}
  \cite{hamana1979injective}
  Every $G$-injective envelope of $\mathcal{S}$ is $G$-rigid.
\end{remark}

\begin{remark}
  Unital completely isometric map is unital completely positive.
\end{remark}

\begin{theorem}
  [Hamana]
  Let $G$ be a  discrete group, and $\mathcal{S}$ be a $G$-oprator system.
  Then, $\mathcal{S}$ has a $G$-injective envelope $(I_G(\mathcal{S}),\kappa)$.
  This injective envelope is uniwue, un the sense that for every $G$-injective envelope $(\mathcal{U},\iota)$ of $\mathcal{S}$, there exists a uc.i. $G$-equivarent map $\varphi : I_G(\mathcal{S}) \rightarrow \mathcal{U}$ s.t. $\varphi \circ \kappa = \iota$. 
\end{theorem}

For an injective $\rm{C}^*$-algebra $\mathcal{A}$, $I_G(\mathcal{A})$ is injective $\rm{C}^*$-algebra w.r.t the Choi-Effros product.

\subsection{Furstenberg boundary}
\begin{definition}
  Let $G$ be a discrete group.
  The Hamana boundary $\d_H G$ of $G$ is the compact space s.t. $I_G(\C) = C(\d_H G)$.
  By contrvariance, the $G$-action on $C(\d_H G)$ induces a $G$-action on $\d_H G$ which we will refer to as the $G$-action on $\d_H G)$.
\end{definition}

\begin{theorem}
  $\d_H G$ is a point if and only if $G$ is amenable.
\end{theorem}

\begin{definition}
  Let $G$ be a group and $X$ be a compact $G$-space.

  The $G$-action on $X$ is minimal if for every $x$ in $X$,
  the $G$-orbit $Gx$ is dense in $X$.

  The $G$-action on $X$ is strongly proximal if for every probability measure $\nu \in \mathcal{P}(X)$,
  the weak* closure of the $G$-orbit $G\nu$ contains a point mass $\delta_x$ for some $x \in X$.

  $X$ is a $G$-boundary if the $G$-action on $X$ is both minimal and strongly proximal,
  i.e. for every probability measure $\nu \in \mathcal{P}(X)$,
 $\overline{G\nu}^{w*} \supset X$.
\end{definition}

\begin{remark}
  The Hamana boundary is a $G$-boundary. 
\end{remark}

\begin{proof}
  ne
\end{proof}

\begin{remark}
  $\d F_2$ is a $F_2$-boundary, but $\d \Z$ is not a $\Z$-boundary.
\end{remark}

\begin{lemma}
  Let $G$ be a group,
  let $M$ be a minimal compact $G$-space and let $B$ be a compact $G$-boundary.
  There is at most one unital positive $G$-equivariant map from $C(B)$ to $C(M)$,
  and if such a map exists, then it is a unital injective $*$-homomorphism.
\end{lemma}

\begin{theorem}
  Let $G$ be a discrete group.
  Then, $\d_F G \cong \d_H G$.
\end{theorem}

\begin{proposition}
  The map taking a probability measure $\nu \in \mathcal{P}(\d_F G)$ to $P_\nu(C(\d_F G))$ is a bijection between $\mathcal{P}(\d_F G)$ and the collection of unital isometoric $G$-equivarinat copies of $C(\d_F G)$ in $l^\infty(G)$.
  The image $P_\nu(C(\d_F G))$ is a $\rm{C}^*$-subalgebra if and only if $\nu$ is a point mass.
\end{proposition}

\begin{theorem}
  Let $G$ be a discrete group.
  Then $G$ is exact if and only if the $G$-action on $\d_F G$ is amenable.
\end{theorem}

\subsection{C*-simplicity}
\begin{theorem}
  Let $G$ be a discrete group.
  There is a canonical nuclear $\rm{C}^*$-algebra $N(C_r^*(G)) = C(\d_F G) \rtimes_r G$ s.t.
  \begin{align*}
    C_r^*(G) \subset N(C_r^*(G) \subset I(C_r^*(G)),
  \end{align*}
  where $I(C_r^*(G))$ denotes the injective envelope of $C_r^*(G)$.
  The algebra $N(C_r^*(G))$ is simple if $C_r^*(G)$ is simple,
  and prime if and only if $C_r^*(G)$ is prime.
\end{theorem}
\begin{definition}
  Let $G$ be adiscrete group, and let $X$ be a compact $G$-space.

  The $G$-action on $X$ is topologically free if for every $s \in G\backslash \{e\}$,
  the set
  \begin{align*}
    X \backslash X^s = \{x\in X|sx \neq x\}
  \end{align*}
  is dense in X.
\end{definition}
\begin{theorem}
  Let $G$ be a discrete group.
  Then the followings are equivalent:
  \begin{enumerate}
  \item The group $G$ is $\rm{C}^*$-simple.
  \item $C(\d_F G) \rtimes_r G$ is simple.
  \item $C(B) \rtimes_r G$ is simple for some $G$-boundary $B$.
  \item The $G$-action on $\d_F G$ is topologically free.
  \item The $G$-action on some $G$-boundary is topologically free.
  \end{enumerate}
\end{theorem}
